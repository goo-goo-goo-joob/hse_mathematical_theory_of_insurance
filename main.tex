%!TEX TS-program = xelatex

\documentclass[a4paper,12pt]{article}

\input{data/preambular.tex}
\begin{document}
\input{data/title.tex}
\tableofcontents
\pagebreak

\section{Постановка задачи}
Для однородной группы из $n = 1000$ клинетов, имеющих вероятность страхового случая $p = 0,05$ и экспоненциальное распределение страховых выплат со средним $5$ денежных единиц, определить значение уровня $k$ для 
\begin{enumerate}
	\item безусловной франшизы
	\item условной франшизы,
\end{enumerate}
которое обеспечивает надежность $\beta = 98,5\%$ при размере собственного капитала $S = 35$. Коэффициент нагрузки равен $\alpha = 11\%$.

\section{Решение}

\subsection{Безусловная франшиза}
Страховые выплаты $X_1^0$ имеют распределение $F_1^0(x) = P\{X_1 \le x | X_1 > 0\}$. Так как $EX_1^0 = 5 = \frac{1}{\lambda}$, то $\lambda = 0,2$. Тогда запишем, что $F_1^0(x) = 1 - \exp(-0,2 x)$.

Определим вероятность ненулевого ущерба (новую вероятность страхового случая):
\[\hat{p} = 1 - \hat{F_1}(0) = 1 - F_1(k) = p (1 - F_1^0(k)) = p \exp(-0,2 k).\]

Зададим $n\hat{p} = np\exp(-0,2k) = 50 \exp (-0,2 k)$. Тогда если $n\hat{p} < 10$, то суммарный риск страховщика приближается сложно-пуассоновской аппроксимацией. Иначе если $n\hat{p} \ge 10$, то распределение суммарного риска страховщика заменяется нормальным распределением.

Найдем, при каких $k$ можно применять сложно-пуассоновскую или нормальную аппроксимацию. Из уравнения $50 \exp (-0,2 k) = 10$ находим, что $k \approx 8,05$. Таким образом, когда $k \le 8,05$, то применяем нормальную аппроксимацию, а в случае, когда $k > 8,05$ - сложно-пуассоновскую.

\subsubsection{Нормальная аппроксимация}
$\hat{X} = \sum_{i=1}^{n} \hat{X}_i$ - суммарный ущерб страховщика. Тогда \[P\{S + (1 + \alpha)\hat{M} \ge \hat{X}\} = \Upphi_{\hat{M}, \hat{\sigma}}(S + (1 + \alpha)\hat{M}) = \Upphi_{0, 1}\left(\cfrac{S + \alpha\hat{M}}{\hat{\sigma}}\right) \ge \beta,\]
где $\hat{M} = E\hat{X} = nE\hat{X}_1 = n\hat{p}E\hat{X}_1^0 = np\exp(-0,2k)E\hat{X}_1^0$. Учитывая, что величина $\hat{X}_1^0$ при условной франшизе также имеет экспоненциальное распределение с параметром $\lambda = 0,2$, запишем, что $\hat{M} = 250\exp(-0,2k)$.
 
Также запишем, что $\hat{\sigma} = \sqrt{n D\hat{X}_1}$, так как ущербы независимы.  Далее распишем $n D\hat{X}_1 = n\hat{p} (D\hat{X}_1^0 + (1-p)E^2\hat{X}_1^0) = \lambda^{-2}np\exp(-0,2k)(2 - p\exp(-0,2k)) = 1250 \exp(-0,2k)(2 - 0,05\exp(-0,2k))$. 

От выражения $\Upphi_{0, 1}\left(\cfrac{35 +0,11\hat{M}}{\hat{\sigma}}\right) \ge 0,985$ перейдем к записи $\cfrac{35 +0,11\hat{M}}{\hat{\sigma}} \ge x_{0,985}$, где $x_{0,985}$ - квантиль стандартного нормального распределения. Далее необходимо решить следующее неравенство относительно $k$ ($k \ge 0$):
\[\cfrac{35 + 0,11 \cdot 250 \exp(-0,2 k)}{\sqrt{1250 \exp(-0,2k)(2 - 0,05\exp(-0,2k))}} \ge 2,17.\]

Получаем, что $k \ge 10,35$. Но для нормальной аппроксимации $k \in [0; 8,05]$. Таким образом, можно сделать вывод, что нет таких $k$, при которых можно применять нормальную аппроксимацию и которые обеспечивают требуемую надежность.

\subsubsection{Сложно-пуассоновская аппроксимация}
Рассмотрим $k > 8,05$, тогда можно применять сложно-пуассоновскую аппроксимацию суммарного ущерба страховщика. 

Распределение $\hat{X}$ приближается формулой 
\[\hat{F}(x) \approx \sum_{j = 0}^{s} \hat{F}_1^{0 * j}(x) \frac{\hat{\lambda}^j}{j!}\exp (-\hat{\lambda}),\]
где $\hat{\lambda} = n\hat{p}$, а $\hat{F}_1^{0 * j}(x)$ - $j$-кратная свертка ($\hat{F}_1^{0 * j}(x) = P\{\hat{X}_1^0 + \dots \hat{X}_j^0 \le x\}$).

Как было отмечено ранее $\hat{F}_1^0(x) = 1 - \exp(-0,2 x)$. Известно, что сумма независимых экспоненциально распределенных случайных величин имеет распределение Эрланга, то есть
\[\hat{F}_1^{0 * j}(x) = 1 - \sum_{i = 0}^{j - 1} \cfrac{(0,2 x)^i}{i!} \exp(-0,2x).\]

Формула из Леммы 1 позволяет определить значение $s$ в формуле для сложно-пуассоновской аппроксимации, чтобы погрешность аппроксимации не превосходила заданную точность (для дальнейших расчетов примем точность, равную $0,01$). Тогда минимальное число членов $s$ для аппроксимации находим из неравенства:
\[\frac{\hat{\lambda}^{s + 1} \min\{1, (s + 1)\exp(-\hat{\lambda})\}}{(s + 1)!} \le 0,01.\]
Так, для каждого значения $k$ должно быть свое значение $s$, чтобы не превосходить заданную погрешность. В работе использовалось значение $s = 14$ для всех $k$. При таком $s$ для любого рассматриваемого $k$ выполнялось неравенство из Леммы 1. То есть использовалось максимальное значение $s$, необходимое, для обеспечения требуемой точности аппроксимации для различных $k$ (для некоторых $k$ значение $s$ для заданной точности могло быть меньше $14$).

С помощью прикладного ПО найдем минимальное значение $k$, при котором выполняется следующее неравенство:
\[\hat{F}(S + (1 + \alpha)\hat{M}) \ge 0,985,\]
где $\hat{F}(x) \approx \sum_{j = 0}^{s} \left(1 - \sum_{i = 0}^{j - 1} \cfrac{(0,2 x)^i}{i!} \exp(-0,2x)\right) \cfrac{\hat{\lambda}^j}{j!}\exp (-\hat{\lambda})$.

Таким образом, с помощью программы на языке Python был получен уровень безусловной франшизы $k=13,36$. На графике (рис. \ref{fig:graph1}) представлена зависимость аппроксимации надежности страховщика $\hat{F}(x)$ от уровня безусловной франшизы $k$ при $s = 14$.

\begin{figure}[H]
	\centering
	\includegraphics[width=0.9\linewidth]{graph1}
	\caption{Надежность страховщика при разных уровнях безусловной франшизы}
	\label{fig:graph1}
\end{figure}

\subsection{Условная франшиза}

\end{document}